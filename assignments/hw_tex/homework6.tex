\documentclass[]{homework}

\begin{document}


\homework{6}{March 11, 2021}

\begin{problem}{1}
  A second-order multi-point method for solving the differential equation
  \[ \frac{d f(x)}{d x} = A(f(x),x) \]
  can be derived using center differencing to approximate the derivative
  instead of using forward differencing, as discussed in class. We will be
  interested in writing down a formula for $f(x+\Delta x)$ given
  knowledge of $f(x)$ and $f(x-\Delta x)$, and then analyzing the
  the stability of this scheme.
  \begin{subproblem}{a}
    Write down an approximation for $f(x+\Delta x)$ using the center differencing formula.
    You can proceed by writing down the formula you derived on the
    midterm and solving for $f(x+\Delta x)$ in terms of $f(x-\Delta x)$ and $A(f(x), x)$.
    For this method, we will therefore need information about the function
    at two points, $f(x-\Delta x)$, and $f(x)$ in order to compute $A(f(x), x)$.
  \end{subproblem}
  \begin{subproblem}{b}
    Derive an equation for the error $\epsilon(x+\Delta x)$ for the multi-point
    method. To do so, use the fact that the numerical solution is only an
    approximation of the true solution, and so includes some error, i.e.
    $f(x) = f_{\rm true}(x) + \epsilon(x)$. Write an equation for how the
    error at one timestep is related to the error at past timesteps.
    You should taylor expand the function $A$ to linear order in $\epsilon$.
    You can then assume that $f_{\rm true}(x)$ obeys the equation you found in part (a)
    to eliminate several terms.
  \end{subproblem}
  \begin{subproblem}{c}
    Determine whether this method is stable when applied to
    decaying, growing, and oscillating differential equations.
    Do so by setting $A(f(x), x) = \lambda f$, with $\lambda <0$,
    $\lambda > 0$, or $\lambda = \pm i \omega$.
    For this method to be stable, we would like to know if there is
    a $\Delta x$ for which the condition
    $|\epsilon(x+\Delta x)| \le |\epsilon(x)|$ will always be true.
    Substitute your expression for $\epsilon(x+\Delta x)$ into this
    inequality.
    Does there exist a (non-zero, positive)
    $\Delta x$ which can guarantee the inequality holds?\\
    \hint{You do not need to know $\epsilon(x)$ or $\epsilon(x-\Delta x)$
    explicitly; it suffices to consider a ``worst-case scenario''.}
  \end{subproblem}

\end{problem}


\begin{problem}{2}
  Complete the Jupyter notebook assignment.
\end{problem}


\end{document}

