\documentclass[]{homework}

\begin{document}

\solutions{9}{April 9, 2021}

\begin{problem}{1}
  \,\\

  \begin{subproblem}{a}
    Each term in the expression has the form $\sin(2\pi \nu t)$, and
    the frequencies correspond to $\nu = 10 {\rm Hz}, 70 {\rm Hz}, 80 {\rm Hz}$.
  \end{subproblem}

  \begin{subproblem}{b}
    Here the maximum (Nyquist) frequency is 80 Hz. We will need to sample at least twice per
    period, or equivalently sample at half the Nyquist freuency to avoid aliasing, so our 
    sampling timestep should be at most $\Delta t = 1/(2 \times 80 {\rm Hz}) = 1/160 s$.
  \end{subproblem}

  \begin{subproblem}{c}
    We need to fully sample at least one period of each signal.
    The lowest frequency signal corresponds to the longest period, so
    we need a sample at least $1/10$ s long.
  \end{subproblem}

\end{problem}

\begin{problem}{2}
  Noting that the signal $g$ will be sampled at $2N$ points, which we will
  label by $n' = 0, 1, ... 2N-1$, we can write the discrete Fourier transform
  of $g$ as
  \[
    g(t_{n'}) = \sum_{m' = 0}^{2N-1} G(\nu_{m'}) e^{2\pi i m' n' / (2N)}\,.
  \]
  We can replace the expression for $G$ with its definition in terms of $H$.
  Noticing that the terms in the sum will be zero when $m' \ge N$, we can
  discard these terms in the sum, and thereby reduce the upper limit in the sum,
  \[
    g(t_{n'}) = \sum_{m' = 0}^{N-1} H(\nu_{m'}) e^{2\pi i m' n' / (2N)}\,.
  \]
  Finally, considering the case where $n' = 2n$ with $n = 0, 1, ... N-1$, 
  and renaming $m' \rightarrow m$, we have
  \[
    g(t_{2n}) = \sum_{m = 0}^{N-1} H(\nu_{m}) e^{2\pi i m n / N}\,.
  \]
  But, this is just the definition of the transform of $h(t_n)$! So we have
  that $g(t_{2n}) = h(t_n)$.

\end{problem}

\end{document}

