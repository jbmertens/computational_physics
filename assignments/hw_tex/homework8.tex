\documentclass[]{homework}

\begin{document}


\homework{8}{March 26, 2021}

\begin{problem}{1}
  In class, we saw two examples of stability analysis for partial
  differential equations: the wave equation, and advection equation.
  Here we will look at the diffusion equation in one time and one
  space dimensions,
  \[
    \partial_t f(t, x) = \partial_x^2 f(t, x)
  \]
  with Eulerian integration
  in time and a centered second derivative. Written explicitly, the
  finite-difference version of this scheme is given by
  \[
    f(t+\Delta t, x) = f(t, x) + \Delta t \frac{f(t, x+\Delta x) - 2 f(t, x) + f(t, x-\Delta x)}{\Delta x^2}\,.
  \]

  \begin{subproblem}{a}
    Given we can write the numerical solution for $f$ as a combination of
    a true and error term, $f = f_{\rm true} + \epsilon_f$, write an
    expression for the error $\epsilon_f(t+\Delta t, x)$ in terms of
    the error at an earlier time, $\epsilon_f(t, x+\Delta x)$, 
    $\epsilon_f(t)$, and $\epsilon_f(t, x-\Delta x)$. You can assume the
    true part of the solution obeys the finite difference equation,
    canceling all of these terms, leaving you with a finite difference
    equation for $\epsilon_f$.
  \end{subproblem}

  \begin{subproblem}{b}
    Assuming the error takes the form
    \[
      \epsilon_f(t, x) = E(t)e^{ikx}\,,
    \]
    substitute this solution into the expression you derived in
    part a, simplify, and solve for $E(t+\Delta t)$.
    A useful identity is
    \[
      e^{i\theta} + e^{-i\theta} = 2 - 4\sin^2(\theta/2)\,.
    \]
  \end{subproblem}

  \begin{subproblem}{c}
    We would like the time-dependent part of the error 
    to satisfy the inequality
    \[
      |E(t+\Delta t)| \le |E(t)|\,.
    \]
    Substitute your expression from part b for $E(t+\Delta t)$.
    Is there any choice of $\Delta t$ and/or $\Delta x$ which
    can guarantee the inequality is satisfied, resulting in a
    stable method?
  \end{subproblem}

\end{problem}


\begin{problem}{2}
  In class last week, we looked at the PDE\_Integration notebook for solving the
  advection equation in one space and one time dimension.
  This notebook can be downloaded from Canvas.

  \begin{subproblem}{a}


    Consider the case of a computational domain with
    {\em periodic} boundary conditions: we would like the
    value of the function at one end of the computational
    domain to match the value at the other.

    In terms of the arrays storing our function, we will need to set
    function values at the boundaries (end) of the array to values
    just inside the boundary at the other end. In other words, we can
    use the conditions\\
    \texttt{
f[0] = f[-2] \# set first value in the array to the second-last\\ 
f[-1] = f[1] \# set the last value in the array to the second
    }

    Implement this in the PDE\_Integration notebook. Integrate for a
    longer time (e.g. up to \texttt{t\_final = 120}) to verify the
    function "wraps around" the box.
  \end{subproblem}

  \begin{subproblem}{b}
    Now consider how to implement periodic boundary conditions in
    2 dimensions. In class, we examined the behavior of the wave
    equation in 2 dimensions; the PDE\_integration notebook has been
    updated to include this code.

    Determine an array slicing condition that can be used to
    implement periodic boundary conditions in this case, and
    so implement periodic boundary conditions in the notebook.
    You may need to integrate for a longer time in order to see the
    wave ``pass through" the boundaries. Setting
    \texttt{t\_final = 130} with \texttt{xs = np.linspace(-100, 100, 50)}
    works well for me.
    Because the integration method we are using is not always stable,
    you may find the integration routine breaks down if you integrate for
    too long, however this should not happen so quickly that you are unable
    to observe the wave passing through the periodic boundaries.
  \end{subproblem}

\end{problem}

\begin{problem}{3}
  Complete the feedback survey on Canvas. This should be available
  early in the week, in the quizzes section of Canvas.
\end{problem}


\end{document}

