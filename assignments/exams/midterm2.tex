\documentclass[]{homework}
\usepackage{amsmath}
\usepackage{wrapfig}
\usepackage{listings}

\usepackage{xcolor}

\definecolor{codegreen}{rgb}{0,0.6,0}
\definecolor{codegray}{rgb}{0.5,0.5,0.5}
\definecolor{codepurple}{rgb}{0.58,0,0.82}
\definecolor{backcolour}{rgb}{0.95,0.95,0.92}

\lstdefinestyle{mystyle}{
    backgroundcolor=\color{backcolour},   
    commentstyle=\color{codegreen},
    keywordstyle=\color{magenta},
    numberstyle=\tiny\color{codegray},
    stringstyle=\color{codepurple},
    basicstyle=\ttfamily\footnotesize,
    breakatwhitespace=false,         
    breaklines=true,                 
    captionpos=b,                    
    keepspaces=true,                 
    numbers=left,                    
    numbersep=5pt,                  
    showspaces=false,                
    showstringspaces=false,
    showtabs=false,                  
    tabsize=2
}

\lstset{style=mystyle}


\begin{document}


\exam{2}{April 2, 2021}

  \noindent\textbf{Read the following.} It contains important information
  about this exam.

  \begin{itemize}
    \item You must submit your solutions
    to Canvas \textbf{by 4:00 pm}! Please plan accordingly.
    This gives you approximately 1.5 hours for this exam.
    \item There is both a written and computational part to this exam. You
    will need to download both from Canvas, and submit solutions for both to Canvas.
    \item You are allowed to use any reference materials you like for this
      exam, except each other.
    \item You should make an effort to answer all questions on this exam.
      Clearly identify your final answers, and clearly explain your solutions.
      This will be graded according to the syllabus rubric, so effort will be
      heavily weighted.
  \end{itemize}


\begin{problem}{1}
  In the computational portion of this exam, we will obtain solutions to the
  Lane-Emden equation, which describes a self-gravitating, spherically symmetric,
  polytropic fluid.  It is often used as a (rather) simple model of a star.
  The differential equation describing this system is given by

  \[
  \frac1{\xi^2} \frac{\mathrm{d}}{\mathrm{d}\xi} \left( \xi^2 \frac{\mathrm{d}}{\mathrm{d}\xi} \theta(\xi) \right) + \theta(\xi)^n = 0.
  \]

  for a coordinate $\xi$ and function $\theta(\xi)$. The variable $n$ is a polytropic
  index which, for our purposes, is just a constant.

  \begin{subproblem}{a}
    Write a solution to this equation when $n = 0$, subject to the conditions
    $\theta(0) = 1$ and $d\theta/d\xi = 0$ when $\xi = 0$.

    \hint{You can directly integrate this system twice, or use a polynomial function as a guess.}
  \end{subproblem}
  \begin{subproblem}{b}
    For solving this system numerically in Python, we need to reduce this equation to a set
    of first-order equations we can supply to \texttt{solve\_ivp}. Provide such a set of
    equations; you will need these for the computational portion of the exam.
  \end{subproblem}

\end{problem}



\begin{problem}{2}
  Clearly explain your answers to the following questions.
  \begin{subproblem}{a}
    For integrating PDEs and ODEs, we saw a number of methods.
    What condition did we require in order for these methods to
    be stable? 
  \end{subproblem}
  \begin{subproblem}{b}
    We were able to develop numerical schemes that were
    accurate to higher order in a resolution $\Delta x$ using
    Richardson Extrapolation. Describe how this is accomplished.
    You may support your description with an example.
  \end{subproblem}
  \begin{subproblem}{c}
    We saw certain partial differential equations could be
    classified according to their hyperbolicity. Describe each of
    these classifications qualitatively, and name one example of a
    system in each case,
    1) elliptic, 2) parabolic, and 3) hyperbolic.

    There are also limitations to this classification system.
    What types of systems might you encounter that are not (readily)
    classified as one of these systems?
  \end{subproblem}
  \begin{subproblem}{d}
    One example of an integration method we encountered was the 
    midpoint method. While we explored this in the context of ODEs,
    this method is also a valid integration scheme for PDEs. 

    The following code integrates the 1+1-dimensional diffusion
    equation by taking an Eulerian timestep. Provide pseudocode, or
    clearly describe, what modifications you could make
    to this code in order to take a step of the midpoint method.

    \note{This problem is intended to be more challenging.
    You do not have to provide working code, although feel free
    to provide code as a solution if you are inclined.}

\begin{lstlisting}[language=Python]
  import numpy as np
  import matplotlib.pyplot as plt

  def diffusive_step(f, dx, dt) :
      """Parameters:
      f: 1-d array containing function values
      dx, dt: point spacing, timestep
      """
      f_new = np.zeros_like(f)

      # Laplacian of f
      f_lap = (f[2:] - 2*f[1:-1] + f[:-2]) / dx**2
      
      # Forward-Euler time step
      f_new[1:-1] = f[1:-1] + dt*f_lap
      
      # Dirichlet boundary conditions
      f_new[0] = 0
      f_new[-1] = 0
      
      return f_new

  xs = np.linspace(-100, 100, 201) # Choose a grid of x-points
  dx = xs[1] - xs[0] # Determine spacing between grid points
  dt = dx/10. # Choose a timestep
  n_steps = 1000 # Number of timesteps to take

  f_ini = np.exp(-xs**2/20**2) # Initial Gaussian profile
  f = f_ini
  for n in range(n_steps) : # Perform integration
      f = diffusive_step(f, dx, dt)
      
  plt.plot(xs, f_ini)
  plt.plot(xs, f)
\end{lstlisting}

  \end{subproblem}
\end{problem}


\end{document}

