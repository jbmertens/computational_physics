\documentclass[]{homework}

\begin{document}

\project

In place of a final exam, we will have a final project assignment. You should either:
\begin{itemize}
\item provide a solution to a challenging original or existing problem, or
\item identify and research a topic you would like to know more about.
\end{itemize}
The proposed work should generally be beyond the scope of assignments so far. The problem or topic should also be connected to physics in principle, although STEM topics more generally are acceptable. It is also acceptable to connect your project to research you are involved in, or a potential career path. In any case, you should pursue a subject you would like to learn more about.

The effort you put into your project should correspond to roughly 3 homeworks.
Group submissions are also allowed.

The project will include two components: code, and a report or presentation.

\begin{itemize}
  \item The code should consist of a detailed, well-documented Jupyter notebook(s) or
well-documented code file(s). The code does not need to be written in Python, although
Python is preferred. There is no length requirement for the code: your goal
here is to provide a solution to an interesting problem, or to develop pedagogical
material related to your chosen topic.
  \item The project should also include one of: 1) a 6-page written report\footnote{Your report should be single spaced, 12 pt font, and can include images, code, equations, and/or citations, although your report should consist predominantly of written text. The 6 page requirement will not be strictly enforced as a maximum or minimum, but you should aim for 6 pages.}, or 2) an appx. 12-minute presentation ({\em preferred})\footnote{Length subject to change depending on the number of presentations. The time will nominally consist of a 12 minute presentation and a few minutes for questions. A pre-recorded presentation is acceptable.}.
The report or presentation should include a discussion of the problem or research topic, including background and motivation. It should also provide a detailed description of the methods you use. Why did you solve the problem the way you did? What numerical techniques are involved in the topic you chose? You should include citations as needed.
\end{itemize}

In case of group submissions, the requirements stand per individual, e.g. a two person group should submit a 12 page report, 24 minute presentation, or both a 6 page report and 12 minute presentation.

On Friday, April 16, individual meetings will be scheduled to discuss project topics.
Projects and presentations will be due May 11; presentations will take the place of the exam period.

\clearpage

\LARGE
{\bf Rubric}

\normalsize
Grading will roughly be based on whether you accomplish the following.

\large
{\bf Presentations}

\normalsize
Prentations should aim for 12 minutes, and papers 6 pages, per person.
For presentations running a minute or two over or behind is fine, but aim for 12.
A couple minutes for any questions will follow each presentation.

Presentation or paper elements should address the following items.

\begin{itemize}
  \item {\bf Introductory material}:\\
  Discussion of background material relevant for context of the problem/topic.\\
  Describe motivation and connection to other areas; why is the topic interesting or important?

  \item {\bf Numerical methodology}:\\
  Introduce any numerical methods you used that were not covered in class.\\
  Describe how you applied numerical techniques to study your topic; discuss any difficulties
  and how you overcame them.

  \item {\bf Project results}:\\
  What results did you find in completing your project?\\
  What did you learn; what questions could you work on in the future?

\end{itemize}

\large
{\bf Notebooks}

\normalsize
A Python notebook (or other code) and any presentation materials should be submitted via canvas.

It should be possible for the instructor (or TA) to run the code and reproduce any results you present.

Again, make sure to document your notebook (code). Between presentation materials and
code, you should attempt to make it easy for someone with little background knowledge in the
area of your project to understand your work.

\large
{\bf Participation}

\normalsize
You should listen actively during others presentations.
Ask at least one question during another presentation.

\end{document}
