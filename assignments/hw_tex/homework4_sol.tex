\documentclass[]{homework}
\usepackage{tikz}
\usepackage{wrapfig}

\begin{document}


\solutions{4}{Feb. 18, 2021}


\begin{problem}{1}


  \begin{subproblem}{a}
    We can minimize the $\chi^2$ value by setting the derivative to zero
    and proceeding to solve for the coefficients $a_i$.
    \[
    \frac{d \chi^2}{d a_j} = 2 \int_0^1 \left( g(x) - \sum_i a_i x^i \right) (-x^j) = 0
    \]
    \[
      \Rightarrow \hspace{1cm} \int_0^1 \left( \sum_i a_i x^{i+j} - x^j g(x) \right) = 0
    \]
    \[
      \Rightarrow \hspace{1cm} \sum_i a_i \int_0^1 x^{i+j} = \int_0^1 x^j g(x)
    \]
    \[
      \Rightarrow \hspace{1cm} \sum_i \frac{1}{i+j+1} a_i = \int_0^1 x^j g(x)
    \]
    This last expression has the form $\mat{A} \vec{a} = \vec{b}$. The components of the
    vector $b$ are given by
    \[
    b_j = \int_0^1 x^j g(x)\,.
    \]
    Once given a particular function $g(x)$, these components can be evaluated.
  \end{subproblem}
  \begin{subproblem}{b}
    We can write out the left-hand side explicitly to gain some insight. The sum
    runs from $i=0$ to a given number $n$. For $n=2$, we will have three equations,
    \begin{align*}
      j=0:& \hspace{1em} a_0 + \frac{1}{2} a_1 + \frac{1}{3} a_2 \\
      j=1:& \hspace{1em} \frac{1}{2} a_0 + \frac{1}{3} a_1 + \frac{1}{4} a_2 \\
      j=2:& \hspace{1em} \frac{1}{3} a_0 + \frac{1}{4} a_1 + \frac{1}{5} a_2\,.
    \end{align*}
    This is of the form $\mat{A} \vec{a}$, where
    \[
      \vec{a} = \begin{pmatrix} a_0 \\ a_1 \\ a_2 \end{pmatrix}, \hspace{1cm}
      \mat{A} = \begin{pmatrix}
        1\,\, \frac{1}{2}\,\, \frac{1}{3} \\
        \frac{1}{2}\,\, \frac{1}{3}\,\, \frac{1}{4}\\
        \frac{1}{3}\,\, \frac{1}{4}\,\, \frac{1}{5}
      \end{pmatrix}\,.
    \]
    For general $n$, the matrix will have components exactly as given by the Hilbert matrix,
    \[
      \mat{A}_{ij} = \frac{1}{i+j+1}\,.
    \]
  \end{subproblem}


\end{problem}


\end{document}

