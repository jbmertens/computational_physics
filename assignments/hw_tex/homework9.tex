\documentclass[]{homework}

\begin{document}


\homework{9}{April 9, 2021}

\begin{problem}{1}
  Consider a signal given by

  \[
  h(t) = \sin(20\pi t) - 3 \sin(140\pi t) + \cos (160 \pi t)
  \]

  for a time $t$ in seconds.

  \begin{subproblem}{a}
    What frequencies $f$ are contained in this signal?
  \end{subproblem}

  \begin{subproblem}{b}
    What is the maximum sampling time step, $\Delta t$, we can use to avoid aliasing if we compute
    a discrete Fourier transform? That is, what is the Nyquist frequency for this signal?
  \end{subproblem}

  \begin{subproblem}{c}
    What is the minimum time interval over which we must sample the signal to fully capture
    all frequencies? That is, how long must we sample the signal for in order to resolve at
    least one full period of all frequencies present?
  \end{subproblem}

\end{problem}


\begin{problem}{2}
  Consider a signal $h(t_n)$ sampled at N values, and a corresponding Fourier transform $H(\nu_m)$.
  There are a number of transformations that can be implemented in the frequency domain.

  One of these is highly analogous to interpolation.
  Suppose we construct a new signal by adding N zeros to the end of $H(\nu_m)$, that is, let 
    \[
      G(\nu_{m'}) = \begin{cases} 
      H(\nu_{m'}), & 0 \le m' < N \\
      0, & N \le m' < 2N\,.
   \end{cases}
    \]
    Show the inverse Fourier transform of $G$ satisfies
    \[
      g(t_{2n}) = h(t_n), \mathrm{\,\,\, where \,\,\,} n = 0, 1, ...\, N-1.
    \]

\end{problem}


\begin{problem}{3}
  Complete the Jupyter notebook portion of the assignment.
\end{problem}

\end{document}

