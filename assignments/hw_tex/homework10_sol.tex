\documentclass[]{homework}

\begin{document}

\solutions{10}

\begin{problem}{1}\footnote{
Note the choice of conventions made in class will result in
an unnormalized discrete Fourier transform and inverse, that is,
an overall normalization factor of $1/N$ is omitted.
} \,

  \begin{subproblem}{a}
    The time-domain signal corresponding to $G(\nu_{m'})$ is given by
    \[
      g(t_{n'}) = \sum_{m' = 0}^{2N-1} G(\nu_{m'}) e^{2\pi i n' m' / (2N)}\,.
    \]
    Because $G(\nu_{m'}) = 0$ for odd $m'$, we can change this to a sum over
    even $m'$ only.
    \[
      g(t_{n'}) = \sum_{m' \in {\rm even}}^{2N-1} G(\nu_{m'}) e^{2\pi i n' m' / (2N)}\,.
    \]
    Letting $m' = 2m$ for a set of integers $m$, we can sum over $m$ from $0$ to $N-1$
    in place of $m'$,
    \[
      g(t_{n'}) = \sum_{m = 0}^{N-1} G(\nu_{2m}) e^{2\pi i n' 2m / (2N)}\,.
    \]
    We can then simplify and substitute in our expression for $H$ in place of $G$,
    \[
      g(t_{n'}) = \sum_{m = 0}^{N-1} H(\nu_{m}) e^{2\pi i n' m / N}\,.
    \]
    But, this is just an expression for the Fourier transform of $h$!
    We still need to be careful and consider the cases where $n' \ge N$ separately,
    but for $n' < N$, we have $g(t_{n'}) = h(t_{n'})$.
    For $n' \ge N$, we have
    \begin{align}
      g(t_{n'}) &= \sum_{m = 0}^{N-1} H(\nu_{m}) e^{2\pi i (n' - N + N) m / N}\nonumber\\
                &= \sum_{m = 0}^{N-1} H(\nu_{m}) e^{2\pi i (n' - N) m / N} e^{2\pi i m}\nonumber\\
                &= \sum_{m = 0}^{N-1} H(\nu_{m}) e^{2\pi i (n' - N) m / N}\nonumber\\
                &= h(t_{n'-N}) \nonumber
    \end{align}
    where we have both added and subtracted $N$ in the exponent in the first line,
    factored out the added $N$ in the second line, then noted
    that $e^{2\pi i m} = 1$ for an integer $m$ per Euler's identity in the third.
    The third line is the Fourier transform of $h$ with the time sample
    taken at $n'-N$.

  \end{subproblem}

  \begin{subproblem}{b}
    For this derivation, it is perhaps easiest to manpulate the final
    expression in terms of $h$ and show its equivalence to the expression
    in terms of $g$. Beginning with the Fourier transform of $h$, we have
    that
    \[
      h(t_{n}) = \sum_{m = 0}^{N-1} H(\nu_{m}) e^{2\pi i n m / N}\,.
    \]
    Evaluating this expression at values of $n$ coincident with $n'+N/2$ gives us
    \begin{align}
      h(t_{n'+N/2}) & = \sum_{m = 0}^{N-1} H(\nu_{m}) e^{2\pi i (n'+N/2) m / N}\nonumber \\
                    & = \sum_{m = 0}^{N-1} H(\nu_{m}) e^{2\pi i n' m / N} e^{\pi i m}\nonumber \\
                    & = \sum_{m = 0}^{N-1} H(\nu_{m}) e^{2\pi i n' m / N} (-1)^m\,.\nonumber 
    \end{align}
    Combining these,
    \[
      h(t_{n'}) + h(t_{n'+N/2}) = \sum_{m = 0}^{N-1} H(\nu_{m}) e^{2\pi i n' m / N} (1 + (-1)^m)\,.
    \]
    The combination $(1 + (-1)^m)$ will be zero for odd $m$, and otherwise $2$, so
    \[
      h(t_{n'}) + h(t_{n'+N/2}) = 2 \sum_{m \in {\rm evens}}^{N-1} H(\nu_{m}) e^{2\pi i n' m / N}\,,
    \]
    which is precisely what we want for $g$, i.e.
    \begin{align}
      g(t_{n'}) &= \sum_{m' = 0}^{N/2-1} G(\nu_{m'}) e^{2\pi i n' m' / (N/2)}\nonumber \\
                &= \sum_{m' = 0}^{N/2-1} G(\nu_{m'}) e^{4\pi i n' m' / N}\nonumber \\
                &= \sum_{m \in {\rm evens}}^{N-1} G(\nu_{m/2}) e^{2\pi i n' m / N}\nonumber \\
                &= \sum_{m \in {\rm evens}}^{N-1} H(\nu_{m}) e^{2\pi i n' m / N}\nonumber
    \end{align}
    for $m = 2m'$. Thus the two sides of the equation in the problem statement are equivalent.
  \end{subproblem}

\end{problem}



\end{document}

